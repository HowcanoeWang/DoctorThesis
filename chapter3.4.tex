\section{Cross-scale image assmilating toolkit}

This section has been modified and published at the ``Remote Sensing" \citep{wang_easyidp_2021}

\begin{center}
  \noindent
  \fbox{
    \begin{minipage}{0.95\textwidth}
    
      \begin{center}
        \textbf{EasyIDP: A Python Package for Intermediate Data Processing in UAV-Based Plant Phenotyping}
      \end{center}

      \noindent Haozhou Wang$^{1}$, Yulin Duan$^{2,3}$, Yun Shi$^{2,3}$, Yoichiro Kato$^{1,4}$, Seishi Ninomiya$^{1,5}$, and Wei Guo$^{1,\star}$
    
      \noindent $^{1}$ International Field Phenomics Research Laboratory, Institute for Sustainable Agro-Ecosystem Services, Graduate School of Agricultural and Life Science, The University of Tokyo, Tokyo 188-0002, Japan;
      
      \noindent $^{2}$ Key Laboratory of Agricultural Remote Sensing, Ministry of Agriculture, Beijing 100081, China;

      \noindent $^{3}$ Institute of Agricultural Resources and Regional Planning, Chinese Academy of Agricultural Sciences, Beijing 100081, China

      \noindent $^{4}$ International Rice Research Institute, Metro Manila 1226, Philippines

      \noindent $^{5}$ Plant Phenomics Research Center, Jiangsu Collaborative Innovation Center for Modern Crop Production, Nanjing Agricultural University, Nanjing 210095, China
      
      \noindent $^{\star}$ Corresponding author

      \begin{spacing}{1.5}
      \textbf{Abstract}
      \end{spacing}
      
      Unmanned aerial vehicle (UAV) and structure from motion (SfM) photogrammetry techniques are widely used for field-based, high-throughput plant phenotyping nowadays, but some of the intermediate processes throughout the workflow remain manual. For example, geographic information system (GIS) software is used to manually assess the 2D/3D field reconstruction quality and cropping region of interests (ROIs) from the whole field. In addition, extracting phenotypic traits from raw UAV images is more competitive than directly from the digital orthomosaic (DOM). Currently, no easy-to-use tools are available to implement previous tasks for commonly used commercial SfM software, such as Pix4D and Agisoft Metashape. Hence, an open source software package called easy intermediate data processor (EasyIDP; MIT license) was developed to decrease the workload in intermediate data processing mentioned above. The functions of the proposed package include (1) an ROI cropping module, assisting in reconstruction quality assessment and cropping ROIs from the whole field, and (2) an ROI reversing module, projecting ROIs to relative raw images. The result showed that both cropping and reversing modules work as expected. Moreover, the effects of ROI height selection and reversed ROI position on raw images to reverse calculation were discussed. This tool shows great potential for decreasing workload in data annotation for machine learning applications.

      \vspace{5mm}
      \textbf{keywords}: orthomosaic; photogrammetry; phenotyping; reverse calculation; Pix4D; Agisoft Metashape; Agisoft PhotoScan

    \end{minipage}
  }
\end{center}

The EasyIDP software paper, strengthing how much effects of this tool has achieved.