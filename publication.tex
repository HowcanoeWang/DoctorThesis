% Add Publications to TOC
\chapter*{Publications}
\addcontentsline{toc}{chapter}{Publications}

\begin{singlespace}
\section*{Journals}

\noindent
Related to the doctor studies

\begin{enumerate}
  \item \textbf{Wang, H.}, Duan, Y., Shi, Y., Kato, Y., Ninomiya, S., Guo, W., 2021a. EasyIDP: A python package for intermediate data processing in UAV-based plant phenotyping. \textit{Remote Sensing} 13, 2622. \url{https://doi.org/10.3390/rs13132622}
  \item \textbf{Wang, H.}, Tang, L., Nishida, E., Fukano, Y., Kato, Y., Guo, W., Drone-based harvest data prediction can reduce on-farm food loss and improve farmer income. \textit{Plant Phenomics}. (Under review)
  \item \textbf{Wang, H.}, Tang, L., Nishida, E., Fukano, Y., Kato, Y., Guo, W., Virtual broccoli farmland by drone-based phenotyping and cross-scale assimilation. (In preparation)
\end{enumerate}

\noindent
Not related to the doctor studies

\begin{enumerate}
  \item Zhang, W., Peng, X., Cui, G., \textbf{Wang, H.}, Takata, D., Guo, W., 2023. Tree Branch Skeleton Extraction from Drone-Based Photogrammetric Point Cloud. \textit{Drones} 7, 65. \url{https://doi.org/10.3390/drones7020065}
  \item Drofova, I., Guo, W., \textbf{Wang, H.}, Adamek, M., 2023. Use of scanning devices for object 3D reconstruction by photogrammetry and visualization in virtual reality. \textit{Bulletin of Electrical Engineering and Informatics} 12, 868–881. \url{https://doi.org/10.11591/eei.v12i2.4584}
  \item Zhao, L., Guo, W., Wang, J., \textbf{Wang, H.}, Duan, Y., Wang, C., Wu, W., Shi, Y., 2021. An Efficient Method for Estimating Wheat Heading Dates Using UAV Images. \textit{Remote Sensing} 13, 3067. \url{https://doi.org/10.3390/rs13163067}
  \item Feldman, A., \textbf{Wang, H}., Fukano, Y., Kato, Y., Ninomiya, S., Guo, W., 2021. EasyDCP: An affordable, high-throughput tool to measure plant phenotypic traits in 3D. \textit{Methods in Ecology and Evolution} 12, 1679–1686. \url{https://doi.org/10.1111/2041-210X.13645}
  
  \item \textbf{Wang, H.}, Yang, T.-R., Waldy, J., Kershaw, J.A., 2021b. Estimating Individual Tree Heights DBHs from Vertically Displaced Spherical Image Pairs. \textit{Mathematical and Computational Forestry \& Natural-Resource Sciences} 13, 1–14. \url{https://mcfns.net/index.php/Journal/article/view/13.1}
  \item Hsu, Y.-H., Kershaw, J.A., Ducey, M.J., Yang, T.-R., \textbf{Wang, H.}, 2021. Sampling with probability proportional to prediction (3P sampling) using covariates derived from spherical images. \textit{Canadian Journal of Forest Research} 51, 1140–1147. \url{https://doi.org/10.1139/cjfr-2020-0498}
  \item Dai, X., Ducey, M.J., Kershaw, J.A., \textbf{Wang, H.}, 2021a. Sector subsampling for basal area ratio estimation: an alternative to big BAF sampling. \textit{Canadian Journal of Forest Research} 1–9. \url{https://doi.org/10.1139/cjfr-2020-0496}
  \item Dai, X., Ducey, M.J., \textbf{Wang, H.}, Yang, T.-R., Hsu, Y.-H., Ogilvie, J., Kershaw, J.A., Jr, 2021b. Biomass estimates derived from sector subsampling of 360° spherical images. \textit{Forestry: An International Journal of Forest Research} 94, 565–575. \url{https://doi.org/10.1093/forestry/cpab023}
  \item \textbf{Wang, H.}, Kershaw, J.A., Yang, T.-R., Hsu, Y.-H., Ma, X., Chen, Y., 2020. An Integrated System for Estimating Forest Basal Area from Spherical Images. \textit{Mathematical and Computational Forestry \& Natural-Resource Sciences} 12, 0–14. \url{http://mcfns.net/index.php/Journal/article/view/12.1}
  \item \textbf{Wang, H.}, Han, D., Mu, Y., Jiang, L., Yao, X., Bai, Y., Lu, Q., Wang, F., 2019. Landscape-level vegetation classification and fractional woody and herbaceous vegetation cover estimation over the dryland ecosystems by unmanned aerial vehicle platform. \textit{Agricultural and Forest Meteorology} 278, 107665. \url{https://doi.org/10.1016/j.agrformet.2019.107665}
\end{enumerate}

\section*{Conferences}

\noindent
Related to the doctor studies

\begin{enumerate}
  \item \textbf{Wang, H.}, Tang, L., Nishida, E., Fukano, Y., Kato, Y., Guo, W. Sept 27-30, 2022. Estimate Optimal Harvest Time by Cross-scale Assimilated Digital Broccoli Farmland (\textbf{poster}), \textit{7th International Plant Phenotyping Symposium: "Plant Phenotyping for a Sustainable Future"}, Wageningen, Netherlands.
  \item \textbf{Wang, H.}, Kato, Y., Guo, W. May 21-22, 2022. Procedural Geometric Modeling for Plant Phenomics by Blender: Case Study of Maize (\textbf{oral}), \textit{農業情報学会JSAI 2022年次大会}, Kyoto, Japan.
  \item \textbf{Wang, H.}, Tang, L., Nishida, E., Fukano, Y., Kato, Y., Guo, W. July 20-22, 2021. Cost-efficient broccoli head phenotyping using aerial imagery and \gls{sfm}-based weakly supervised learning (\textbf{poster}), \textit{The 8th International Horticulture Research Conference}, Nanjing, Jiangsu, China.
  \item \textbf{Wang, H.}, Kato, Y., Guo, W., June 3-4, 2021. EasyIDP: A python package for intermediate data processing in UAV based plant phenotyping (\textbf{poster}), \textit{超分野植物科学研究会の第1回研究集会 2021}, Zoom online, Tokyo, Japan.
  \item \textbf{Wang, H.}, Kato, Y., Guo, W., May 22, 2021. EasyIDP: A python package for intermediate data processing in UAV based plant phenotyping (\textbf{poster}), \textit{農業情報学会JSAI 2021年次大会}, Zoom online, Tokyo, Japan.
  \item Feldman, A., \textbf{Wang, H.}, Fukano, Y., Kato, Y., Ninomiya, S., Guo, W., February 24-27, 2020. Affordable high-throughput processing of handheld camera images of container plants to phenotypic data (\textbf{poster}), \textit{Phenome 2020}, Tucson Convention Center, Tucson, Arizona, U.S.
  \item Feldman, A., \textbf{Wang, H.}, Fukano, Y., Guo, W., October 22-25, 2019. Affordable high-throughput processing of multi-scale images to phenotypic data (\textbf{poster}). \textit{The 6th International Plant Phenotyping Symposium}, Nanjing, Jiangsu, China.
\end{enumerate}

\noindent
Not related to the doctor studies

\begin{enumerate}
  \item \textbf{Wang, H.}, Kershaw, J.A., June 23-25, 2019. Estimating Forest Attributes from Spherical Images (\textbf{oral}, \textbf{poster}). \textit{The Western Mensurationists 2019 Annual Meeting}, Kamloops Hotel and Conf. Center, Kamloops, British Columbia, Canada.
  \item \textbf{Wang, H.},, Kershaw, J.A., March 23, 2018. Measuring Plant Area Index (PAI) from panorama photo images (\textbf{oral}). \textit{The 25th Annual UNB Graduate Research Conference (GRC)}, Wu Conference Center, Fredericton, New Brunswick, Canada.
  \item \textbf{Wang, H.}, Kershaw, J.A., November 5-7, 2017. Extracting DBH Measurements from RGB Photo Images (\textbf{oral}). \textit{The Northeastern Mensurationists 2017 Annual Meeting}, The Inn at Saratoga. Saratoga Springs, New York, U.S.
  \item \textbf{Wang, H.}, Wang, F., Yao, X., Mu, Y., Bai, Y., Lu, Q., August 20-25, 2017 . UAV-HiRAP: A novel method to improve landscape-level vegetation classification and coverage fraction estimation with unmanned aerial vehicle platform (\textbf{oral}). \textit{The 12th International Congress of Ecological (INTECOL)}, China National Convention Center, Beijing, China.
\end{enumerate}

\section*{Source Code}

\noindent
Easy-Series Software packages

\begin{enumerate}
  \item EastAMS - A GUI plugin tool for Agisoft Metashape with extended functions for smart agriculture. \url{https://github.com/UTokyo-FieldPhenomics-Lab/EasyAMS}
  \item EastBPY - The wrapper for blender python API without pip install extra packages. \url{https://github.com/UTokyo-FieldPhenomics-Lab/EasyBPY}
  \item EasyIDP - A handy tool for dealing with region of interest (ROI) on the image reconstruction (Metashape \& Pix4D) outputs, mainly in agriculture applications. \url{https://github.com/UTokyo-FieldPhenomics-Lab/EasyIDP}
  \item  EasyDCP - Easy Dense Cloud Phenotyping. \url{https://github.com/UTokyo-FieldPhenomics-Lab/EasyDCP}
\end{enumerate}

\noindent
Projects

\begin{enumerate}
  \item UAVbroccoli - Protocol for broccoli data analysis by Metashape. \url{https://github.com/UTokyo-FieldPhenomics-Lab/UAVbroccoli}
  \item Foldio360\_3D\_Reconstruct\_Platform - a protocol for easy indooor 3D reconstruction for small objects. \url{https://github.com/UTokyo-FieldPhenomics-Lab/Foldio360_3D_Reconstruct_Platform}
\end{enumerate}

\noindent
Website maintainer

\begin{enumerate}
  \item \url{www.global-wheat.com} - The website for global wheat dataset. \url{https://github.com/UTokyo-FieldPhenomics-Lab/global-wheat.github.io}
  \item \url{2023.mlcas.site} - The website for Fifth International Workshop on Machine Learning for Cyber-Agricultural Systems (MLCAS2023). \url{https://github.com/UTokyo-FieldPhenomics-Lab/mlcas2023.github.io}
  \item \url{lab.fieldphenomics.com} - The official website for Laboratory of Field Phenomics at UTokyo. \url{https://github.com/UTokyo-FieldPhenomics-Lab/utokyo-fieldphenomics-lab.github.io}
  \item \url{mlcas2021.github.io} - The website for Third International Workshop on Machine Learning for Cyber-Agricultural Systems (MLCAS2021). \url{https://github.com/mlcas2021/mlcas2021.github.io}
\end{enumerate}

\end{singlespace}