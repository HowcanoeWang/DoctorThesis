\subsection{Backward projection methodology}
\label{spp:backward}

% \input{block/easyidp.tex}

The function of this backward projection is projecting ROI from world coordinate to relative UAV raw images. In this section, the external and internal camera parameters generated from SfM-MVS reconstruction projects were loaded by two internal Python packages, ``zipfile| and ``xml". Then the calculation algorithms driven by the pinhole camera model and camera distortion calibration were introduced.

\subsubsection*{Camera parameters loading}

The relationship between field and UAV raw images is built after running SfM-MVS software. It has two main parts, external and internal parameters. The external parameters are different for each raw image, including the camera position (x, y, z) in the real-world coordinate ($O_{world}$, Fig.~\ref{fig:idps1}.a) and the camera rotation (yaw, pitch, roll). The internal parameters describe the characteristics of the sensor and are the same as each raw image, such as focal length, camera Charge-coupled Device (CCD) size, and lens distortion calibration parameters. These parameters are available under the Agisfot Metashape and Pix4D project intermediate files.

\subsubsection*{Backward projection forumulars}

The geometry from the real-world coordinate ($O_{world}$) to image pixel coordinate ($O_{pix}$) is shown in Fig. \ref{fig:idps1}.a-c. There are four coordinate systems, first is the $O_{world}$ whose unit is often meter (Fig. \ref{fig:idps1}.a). The second one is the camera coordinate ($O_{cam}$, Fig. \ref{fig:idps1}.b), which makes the camera position to the origin (0,0,0) of coordinates, and the camera optical axis is used as the z-axis (commonly, the point $O_{img}$ is not the center point of plane). The third is the camera CCD coordinate ($O_{img}$, Fig. \ref{fig:idps1}.c) whose unit is often mm. The last one is pixel coordinate ($O_{pix}$) whose origin is the top left corner in the $O_{img}$ and the unit is pixel.

Let us assume a point $P_{world} (x_w,y_w,z_w)$ in $O_{world}$, to transform that point into $P_{cam} (x_c,y_c,z_c)$ in $O_{cam}$ (Fig. \ref{fig:idps1}.a), the $4\times4$ transform matrix $T$ could be derived from camera position ($t$, translational transformation) and camera rotation ($R$, rotational transformation):

\begin{equation}
  \begin{array}{lll}
    P_{cam} & = & T \cdot P_{world} \\
    \begin{small}
      \left[ 
        % amsmath package is required
        \begin{matrix} 
          x_c \\
          y_c \\
          z_c \\
          1
        \end{matrix} 
      \right] 
    \end{small}
    & = & 
      \begin{small}
      \left[ 
        % amsmath package is required
        \begin{matrix} 
          R_{11} & R_{12} & R_{13} & t_1 \\
          R_{11} & R_{12} & R_{13} & t_1  \\
          R_{11} & R_{12} & R_{13} & t_1  \\
          0      & 0      & 0      & 1
        \end{matrix} 
      \right] 
      \left[ 
        % amsmath package is required
        \begin{matrix} 
          x_w \\
          y_w \\
          z_w \\
          1
        \end{matrix} 
      \right] 
    \end{small}
  \end{array}
\label{eq:idp1}
\end{equation}

\noindent 
Where, $t$ is the $3\times1$ position matrix, and $R$ is the $3\times3$ rotation matrix derived by $(\omega, \varphi, \kappa)$ from camera rotation parameters (yaw, pitch, roll): \ref{eq:idp2}

\begin{equation}
  \begin{array}{lll}
    R & = & R_{x}(\omega) R_{y}(\varphi) R_{z}(\kappa) \\
    & = &
      \begin{small}
      \left[ 
        % amsmath package is required
        \begin{matrix} 
          1 & 0           & 0 \\
          0 & cos(\omega) & -sin(\omega) \\
          0 & sin(\omega) & cos(\omega)
        \end{matrix} 
      \right] 
      \left[ 
        \begin{matrix} 
          cos(\varphi)  & 0 & sin(\varphi) \\
          0             & 1 & 0 \\
          -sin(\varphi) & 0 & cos(\varphi)
        \end{matrix} 
      \right] 
      \left[ 
        \begin{matrix} 
          cos(\kappa) & -sin(\kappa) & 0 \\
          sin(\kappa) & cos(\kappa)  & 0 \\
          0           & 0            & 1
        \end{matrix} 
      \right] 
      \end{small} \\
    & = &
      \begin{small}
        \left[ 
          \begin{matrix} 
            cos \kappa cos \varphi  
              & -sin \kappa cos \varphi  
              & sin \varphi  \\
        
            cos \kappa sin \omega sin \varphi  + sin \kappa cos \omega 
              & cos \kappa cos \omega  - sin \kappa sin \omega sin \varphi 
              & -sin \omega cos \varphi  \\
        
            sin \kappa sin \omega  - cos \kappa cos \omega sin \varphi 
              & sin \kappa cos \omega sin \varphi  + cos \kappa sin \omega  
              & cos \omega cos \varphi 
          \end{matrix} 
        \right] 
      \end{small}
  \end{array}
  \label{eq:idp2}
\end{equation}

\begin{figure}[htb]
  \begin{center}
    \resizebox{\textwidth}{!}{
      \includegraphics{figures/idp/Fig.S1_world2pixels.pdf}
    }
  \end{center}
  \caption[Backward projection illustration graph]{
     Backward projecting one point from the 3D world coordinates to the 2D pixel coordinates on raw UAV images by pinhole camera model. (a) The relationship between world coordinate ($O_{world}$) and camera coordinate ($O_{cam}$), linked by camera external parameters (position and rotation). (b) the relationship between camera coordinate ($O_{cam}$) and image coordinate ($O_[img]$). (c) The relationship between image coordinate ($O_[img]$) and pixel coordinate ($O_{pix}$). (d) The camera distortion calibration between undistorted images and distorted images caused by the lens.
  }
  \label{fig:idps1}
\end{figure}