\subsection{Backward projection methodology}
\label{spp:backward}

% \input{block/easyidp.tex}

The function of this backward projection is projecting ROI from world coordinate to relative UAV raw images. In this section, using ``zipfile" and ``xml" Python packages to load the external and internal camera parameters generated from SfM-MVS reconstruction projects. Then apply the backward calculation using the pinhole camera model. Finally camera distortion calibration is conducted.

\subsubsection{Camera parameters loading}

The relationship between field and UAV raw images is built after running SfM-MVS software. It has two main parts, external and internal parameters. The external parameters are different for each raw image, including the camera position (x, y, z) in the real-world coordinate ($O_{world}$, Fig.~\ref{fig:idps1}.a) and the camera rotation (yaw, pitch, roll). The internal parameters describe the characteristics of the sensor and are the same as each raw image, such as focal length, camera Charge-coupled Device (CCD) size, and lens distortion calibration parameters (\url{https://support.pix4d.com/hc/en-us/articles/202559089-How-are-the-Internal-and-External-Camera-Parameters-defined}). These parameters are available under the Agisfot Metashape and Pix4D project intermediate files.

For Pix4D projects, all these parameters are located in the ``1\_initial/params" folder, the ``calibrated internal camera parameters.cam", ``calibrated camera paramters.txt", ``pmatrix.txt", and ``offset.xyz" are loaded as text directly and parsed in the EasyIDP package without any external packages. For more details about those files, please refer to the Pix4D official documentation (\url{https://support.pix4d.com/hc/en-us/articles/202559089-How-are-the-Internal-and-External-Camera-Parameters-defined}).

For Agisoft Metashape projects, all these parameters can be obtained either by calling APIs (Professional license required) or by reading zipped xml files in the project file ``project.files/0/chunks.zip/doc.xml". The EasyIDP package chose the zipped xml way without a professional license. The ``zipfile" and ``xml" packages were used to un-zip and parse these parameters in xml files.

\begin{figure}[htb]
  \begin{center}
    \resizebox{\textwidth}{!}{
      \includegraphics{figures/idp/Fig.S1_world2pixels.pdf}
    }
  \end{center}
  \caption[Backward projection illustration graph]{
     Backward projecting one point from the 3D world coordinates to the 2D pixel coordinates on raw UAV images by pinhole camera model. (a) The relationship between world coordinate ($O_{world}$) and camera coordinate ($O_{cam}$), linked by camera external parameters (position and rotation). (b) the relationship between camera coordinate ($O_{cam}$) and image coordinate ($O_[img]$). (c) The relationship between image coordinate ($O_[img]$) and pixel coordinate ($O_{pix}$). (d) The camera distortion calibration between undistorted images and distorted images caused by the lens.
  }
  \label{fig:idps1}
\end{figure}

\subsubsection{Backward projection formulas}

The geometry from the real-world coordinate ($O_{world}$) to image pixel coordinate ($O_{pix}$) is shown in Fig. \ref{fig:idps1}.a-c. There are four coordinate systems, first is the $O_{world}$ whose unit is often meter (Fig. \ref{fig:idps1}.a). The second one is the camera coordinate ($O_{cam}$, Fig. \ref{fig:idps1}.b), which makes the camera position to the origin (0,0,0) of coordinates, and the camera optical axis is used as the z-axis (commonly, the point $O_{img}$ is not the center point of plane). The third is the camera CCD coordinate ($O_{img}$, Fig.~\ref{fig:idps1}.c) whose unit is often mm. The last one is pixel coordinate ($O_{pix}$) whose origin is the top left corner in the $O_{img}$ and the unit is pixel.

Let us assume a point $P_{world} (x_w,y_w,z_w)$ in $O_{world}$, to transform that point into $P_{cam} (x_c,y_c,z_c)$ in $O_{cam}$ (Fig.~\ref{fig:idps1}.a), the $3\times4$ transform matrix $T$ could be derived from camera position ($t$, translational transformation) and camera rotation ($R$, rotational transformation):

\begin{equation}
  \begin{array}{lll}
    P_{cam} & = & T \cdot P_{world} \\
    \begin{small}
      \left[ 
        % amsmath package is required
        \begin{matrix} 
          x_c \\
          y_c \\
          z_c \\
          1
        \end{matrix} 
      \right] 
    \end{small}
    & = & 
      \begin{small}
      \left[ 
        % amsmath package is required
        \begin{matrix} 
          R_{11} & R_{12} & R_{13} & t_1 \\
          R_{11} & R_{12} & R_{13} & t_1  \\
          R_{11} & R_{12} & R_{13} & t_1  \\
          0      & 0      & 0      & 1
        \end{matrix} 
      \right] 
      \left[ 
        % amsmath package is required
        \begin{matrix} 
          x_w \\
          y_w \\
          z_w \\
          1
        \end{matrix} 
      \right] 
    \end{small}
  \end{array}
\label{eq:idp1}
\end{equation}

\noindent 
Where, $t$ is the $3\times1$ position matrix, and $R$ is the $3\times3$ rotation matrix derived by $(\omega, \varphi, \kappa)$ from camera rotation parameters (yaw, pitch, roll) (\url{https://support.pix4d.com/hc/en-us/articles/202558969-Yaw-Pitch-Roll-and-Omega-Phi-Kappa-angles}):

% \begin{equation}
  \begin{array}{lll}
    R & = & R_{x}(\omega) R_{y}(\varphi) R_{z}(\kappa) \\
    & = &
      \begin{small}
      \left[ 
        % amsmath package is required
        \begin{matrix} 
          1 & 0           & 0 \\
          0 & cos(\omega) & -sin(\omega) \\
          0 & sin(\omega) & cos(\omega)
        \end{matrix} 
      \right] 
      \left[ 
        \begin{matrix} 
          cos(\varphi)  & 0 & sin(\varphi) \\
          0             & 1 & 0 \\
          -sin(\varphi) & 0 & cos(\varphi)
        \end{matrix} 
      \right] 
      \left[ 
        \begin{matrix} 
          cos(\kappa) & -sin(\kappa) & 0 \\
          sin(\kappa) & cos(\kappa)  & 0 \\
          0           & 0            & 1
        \end{matrix} 
      \right] 
      \end{small} \\
    & = &
      \begin{small}
        \left[ 
          \begin{matrix} 
            cos \kappa cos \varphi  
              & -sin \kappa cos \varphi  
              & sin \varphi  \\
        
            cos \kappa sin \omega sin \varphi  + sin \kappa cos \omega 
              & cos \kappa cos \omega  - sin \kappa sin \omega sin \varphi 
              & -sin \omega cos \varphi  \\
        
            sin \kappa sin \omega  - cos \kappa cos \omega sin \varphi 
              & sin \kappa cos \omega sin \varphi  + cos \kappa sin \omega  
              & cos \omega cos \varphi 
          \end{matrix} 
        \right] 
      \end{small}
  \end{array}
  \label{eq:idp2}
\end{equation}

The distance from the normalized plane to the origin $O_{cam}$  is 1 mm while the distance from the camera CCD plane to the origin $O_{cam}$ is focal length $f$ (Fig.~\ref{fig:idps1}.b) in mm. The transformation from $P_{cam}(x_c,y_c,z_c)$ to normalized plane $P_{norm} (x_n,y_n)$ and camera CCD plane $P_{img} (x_i,y_i)$ can be derived by triangle similarity:

\begin{equation}
  \begin{bmatrix} 
    x_i \\ y_i \\ 1
  \end{bmatrix} 
  = f
    \begin{bmatrix} 
      x_n \\ y_n \\ 1
    \end{bmatrix} 
  = f
    \begin{bmatrix} 
      \frac{x_c}{z_c} \\
      \frac{y_c}{z_c} \\
      1
    \end{bmatrix} 
  = \frac{f}{z_c}
    \begin{bmatrix} 
      x_c \\ y_c \\ z_c
    \end{bmatrix} 
\label{eq:idp3}
\end{equation}

To transform $P_{img} (x_i,y_i)$ in mm to the image pixel coordinate position $P_{pix} (x_p,y_p)$ in pixel (Fig.~\ref{fig:idps1}.c), the following set of equations should be applied:

\begin{equation}
  \left\{
  \begin{array}{lllllll}
    x_p & = & \alpha \cdot x_i + c_x 
        & = & f \cdot \alpha \cdot x_n + c_x 
        & = & f_{\alpha} \cdot x_n + c_x \\
    y_p & = & \beta \cdot y_i + c_y 
        & = & f \cdot \beta \cdot y_n + c_y 
        & = & f_{\beta} \cdot y_x + c_y
  \end{array}
  \right.
\label{eq:idp4}
\end{equation}

\noindent 
Where, $\alpha$ and $\beta$ are the pixel resolution whose unit is pixel/mm, and often are the same in pinhole camera models. $f_{\alpha}$ and $f_{\beta}$ is the focal length in pixel. Notably, Pix4D $(c_x,c_y)$ can be obtained directly, while for Agisoft Metashape (\url{https://www.agisoft.com/pdf/metashape-pro_1_7_en.pdf.} p. 176), $(c_x,c_y)$ in the xml file is not what is defined here, it is the offset to image center, which actually equals to $(0.5w+c_x,0.5h+c_y)$, where $w$ and $h$ are the pixel width and pixel height, respectively.

Equations \eqref{eq:idp4} can be expressed in the following homogeneous coordinate form: 

\begin{equation}
  \begin{bmatrix}
    x_p \\ y_p \\ 1 
  \end{bmatrix}
  =
  \begin{bmatrix}
    f_{\alpha} & 0         & c_x \\
    0          & f_{\beta} & c_y \\
    0          & 0         & 1
  \end{bmatrix}
  \begin{bmatrix}
    x_n \\ y_n \\ 1
  \end{bmatrix}
  = K
  \begin{bmatrix}
    x_n \\ y_n \\ 1
  \end{bmatrix}
\label{eq:idp5}
\end{equation}

Sum up equations \eqref{eq:idp1} to \eqref{eq:idp5}; to transform $P_w (x_w,y_w,z_w)$ directly to $P_{pix} (x_p,y_p)$:

\begin{equation}
  \begin{bmatrix}
    x_p \\ y_p \\ 1 
  \end{bmatrix}
  = K
    \begin{bmatrix}
      x_n \\ y_n \\ 1
    \end{bmatrix}
  = \frac{1}{z_c} K 
    \begin{bmatrix}
      x_c \\ y_c \\ 1
    \end{bmatrix}
  = \frac{1}{z_c} 
    \begin{bmatrix}
      K & 1
    \end{bmatrix}
    T
    \begin{bmatrix}
      x_w \\ y_w \\ z_w \\ 1
    \end{bmatrix}
  = P_{mat}
    \begin{bmatrix}
      x_w \\ y_w \\ z_w \\ 1
    \end{bmatrix}
\label{eq:idp6}
\end{equation}

\noindent
where the $3\times4$ matrix $P_{mat}$ is often called the projection matrix, which can directly transform points in 3D world coordinates to 2D pixel coordinates.

\subsubsection{Camera distortion calibration}

The equation \eqref{eq:idp5} transformation is idealized, and the distortion caused by the camera lens is neglected (Fig.~\ref{fig:idps1}.d). Several camera calibration parameters are used to correct this distortion, including three or four radial distortion coefficients ($K_i$ in MetaShape and $R_i$ in Pix4D) and two tangential distortion coefficients ($P_i$ in MetaShape and $T_i$ in Pix4D). Metashape sometimes provides affinity ($B_1$) and non-orthogonality ($B_2$) coefficients in pixels. The correction equation to distorted pixel position $(x_p^\prime,y_p^\prime)$ are as follows:

\begin{equation}
  \begin{array}{lll}
  \text{Pix4D} & = &
    \begin{cases}
      x_p^\prime =  c_x + x^\prime f \\
      y_p^\prime =  c_y + y^\prime f 
    \end{cases}
  
  \\ \\

  \text{Metashape} & = &
    \begin{cases}
      x_p^\prime = c_x + x^\prime f + x^\prime B_1 + y^\prime B_2 \\
      y_p^\prime = c_y + y^\prime f 
    \end{cases}

  \end{array}
\label{eq:idp7}
\end{equation}

\noindent
where:

\begin{equation}
  \begin{array}{lll}
    \text{Pix4D} & = &
    \begin{cases}
      x^\prime = k_0 x_n + 2 T_2 x_n y_n + T_1(r^2 + 2 x_n^2) \\
      y^\prime = k_0 y_n + 2 T_1 x_n y_n + T_2(r^2 + 2 y_n^2)
    \end{cases}
  
  \\ \\

  \text{Metashape} & = &
    \begin{cases}
      x^\prime = k_0 x_n + 2 P_2 x_n y_n + P_1(r^2 + 2 x_n^2) \\
      y^\prime = k_0 y_n + 2 P_1 x_n y_n + P_2(r^2 + 2 y_n^2)
    \end{cases}
  \end{array}
\label{eq:idp8}
\end{equation}

\noindent
and: 

\begin{equation}
  \begin{array}{llll}
    r   & = & \sqrt{x_n^2 + y_n^2} & \\ \\
    k_0 & = & 
      \begin{cases}
        1 + R_1 r^2 + R_2 r^4 + R_3 r^6           & (\text{Pix4D}) \\
        1 + K_1 r^2 + K_2 r^4 + K_3 r^6 + K_4 r^8 & (\text{Metashape})\\
      \end{cases}
  \end{array}
\label{eq:idp9}
\end{equation}