\chapter{Destructive 3D phenotyping pipeline}

\section{Introduction}

Estimating the plant phenotypes accurately and efficiently can help to bridge the gap between genotype and phenotype. The traditional phenotyping measurement is time-consuming, laborious, and often not accurate. Although several authors have developed 2D image-based phenotyping methods which are more efficient, non-destructive, and have higher throughput \citep{yang_greenness_2015,guo_easypcc_2017,zou_broccoli_2019}, these approaches are unable to describe the plant 3D structure due to the occlusion and dimension loss when projecting onto the 2D plane. As a result, it produces inaccuracies and uncertainties for advanced phenotyping applications.

To overcome the drawbacks of 2D image-based phenotyping, several studies have paid attention to 3D approaches. \citet{paulus_measuring_2019} and \citet{kochi_introduction_2021} have summarized the current approaches to obtain 3D plant models, and a large number of studies have chosen the 3D reconstruction by photogrammetry using common RGB cameras due to the low device cost \citep{xiao_estimating_2021,zermas_3d_2020,zhang_estimating_2016}. The key idea of sfm, was taking images from different angle views and calcualting their relative positions to object.

% the key idea of sfm, and the classes of sfm


\begin{figure}[htb]
  \begin{center}
    \resizebox{\textwidth}{!}{
      \includegraphics{figures/des/sfm_types.pdf}
    }
  \end{center}
  \caption[Current photogrammetry (3D reconstruction) methods]{
    The current camera placement for 3D reconstruction. (a) fixing the object and taking images using multiple fixed cameras at the same time, also called forward intersection; (b) fixing the object but taking images by using a moved camera, also called backward resection; (c) rotating the object and taking images using fewer multiple fixed cameras, or a camera fixed at different locations for each rotation.
  }
  \label{fig:des1}
\end{figure}


\section{Methods and Materials}

\subsection{Plant 3D model acquisition}

% plant materials, the information for fields and selective sampling

\subsubsection{Imaging device}

% figure: 3D reconstruction devices

\subsubsection{Image preprocessing}



\subsubsection{Batch 3D reconstruction}

% marker detection for one round and ignore for others

% the batch preprocessing 

\subsection{Phenotypic traits extraction}

% figure: general workflow

\subsubsection{Head segmentation}

\subsubsection{Upward direction correction}

\subsubsection{Traits calculation}


\subsection{Validation}

% field measurmenet

% r2 and rmse


\section{Results}

\subsection{Image preprocessing}

% figure: unet + casadePSP results

\subsection{Plant 3D model}

% figure: comparison between real photo and photos + model 3 views

\subsection{Traits extraction}

% figure: head extraction

% figure: traits 3D view

\subsection{Validation}

% figure: method_compare.pdf



\section{Discussion}



\section{Conclusion}