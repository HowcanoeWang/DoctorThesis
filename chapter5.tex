\chapter{General Conclusion}

\section{Overall conclusions}

\section{Future research prospects}

% Virtual plants powered crop data analysis and phenotyping applications.

% The plant and its canopy architecture influence the responses and interactions with various environmental factors, which ultimately affects the yield. The conventional analysis methods rely on heavy and repetitive field measured geometry traits along with statistical analysis through whole growing seasons. While the advent of digital twin technology points a very different future, the researchers can operate directly on the virtual plants and preview its impact on the plant immediately \citet{verdouw_digital_2021}. The fundamental in implementing such technology is to present the 3D crop model (virtual plant) accurately in the computer first, and then implement architecture fine-tuning and phenotyping applications on them. This seminar will summarize recent published papers on how this has been achieved.

% Currently, there are three main methods of acquiring virtual plant that can be architecture fine-tuned. 1) By template stitching. chang_3dcap_2022 obtained the database of wheat organ 3D models at different growth stages by destructive sampling and manual measurement of parameters. Then combine those organed to one wheat by random picking and placing with random position angles within the range obtained by real measurements. wen_3d_2021 applied the similar idea for maize. 2) By static deformation. The 3D static model of whole maize (in point cloud format) was obtained by 3D reconstruction. The maize model was then automatically segmented to organs, and these segmented parts were transformed by applying geometric transformations to implement changes in architecture liu_canopy_2021. 3) By parametric approximation. The L-system was used to construct a parameter adjustable virtual crop model and then adjusted the parameters to approximate the real crop by three-view photos cieslak_l-system_2021. This process was also reflected in the implementation of CG modelling. For example, the general shape of the plant was obtained through relatively simple geometries, and then photo-quality texture mapping was used to obtain a very realistic CG model mikami_hidden_2022.

% Several phenotyping applications can be simulated on the adjustable virtual 3D plant models. 1) The ray-tracing technologies can be applied to simulate lights within the canopy. For example, analysis of the contributions of foliar and nonfoliar tissues chang_3dcap_2022, leaf inclination angles liu_canopy_2021, and row spacing he_modeling_2021 to canopy photosynthesis efficiency of each crop. 2) The LiDAR point cloud simulation technology can be used to simulate point clouds from the virtual canopy. For the canopy traits which are easy to get from virtual plants but hard from the real field, the corresponding models between features from simulated point cloud and traits from virtual canopy can be trained, and then applied on the field scanned LiDAR data to inverse the actual data liu_estimating_2017. 3) the CG rendering techniques can decrease the workload of data annotating in the phenotyping data processing by deep learning. The rending engine can yield CG photos that are very close to the real world. At the same time, by adjusting the model texture to solid pure color, the corresponding labelled information for both CG photos mikami_hidden_2022 and point clouds chaudhury_3d_2020 can be generated in batch from random virtual canopies.
