\begin{table}[htb]
  \caption[Softwares for photogrammetry 3D reconstruction]{Softwares for photogrammetry 3D reconstruction. ``Close-range'' is mainly used for building 3D models for individual plants or organs without \gls{gps} coordinates; ``Aerial'' is mainly used for building geo-referenced canopy 3D models using the UAV imagery with \gls{gps} coordinates.}
  \label{tbl:int1}
  % \begin{adjustwidth}{-0.05\textwidth}{-0.05\textwidth}
    \begin{center}
    \resizebox{0.9\textwidth}{!}{
      \begin{threeparttable}
      \begin{tabular}{llcc}
        \hline
        \multicolumn{1}{c}{\textbf{Types}} & \multicolumn{1}{c}{\textbf{Software name}}                                                                  & \textbf{close-range} & \textbf{aerial} \\ \hline
        Open Source                        & AliceVision Meshrooms                                                                                       & $\checkmark$         & $\checkmark$    \\
                                           & COLMAP                                                                                                      & $\checkmark$         & $\checkmark$    \\
                                           & Multi-View Environment (MVE)                                                                                & $\checkmark$         & $\times$        \\
                                           & OpenDroneMap\tnote{1}                                                                                       & $\times$             & $\checkmark$    \\
                                           & OpenMVG $\rightarrow \begin{cases}\text{OpenMVS} \\ \text{MVE} \end{cases}$                                 & $\checkmark$         & $\checkmark$    \\
                                           & VisualSFM\tnote{2} $ \rightarrow \begin{cases}\text{MeshRecon}\\ \text{PMVS} \\ \text{OpenMVS} \end{cases}$ & $\checkmark$         & $?$\tnote{2}      \\
        Commercial                         & 3DF Zephyr                                                                                                  & $\checkmark$         & $\checkmark$    \\
                                           & Agisoft Metashape                                                                                           & $\checkmark$         & $\checkmark$    \\
                                           & ContextCapture                                                                                              & $\checkmark$         & $\checkmark$    \\
                                           & Correlator3D                                                                                                & $\times$             & $\checkmark$    \\
                                           & DroneDeploy                                                                                                 & $\times$             & $\checkmark$    \\
                                           & Elcocision 10                                                                                               & $\checkmark$         & $\checkmark$    \\
                                           & Pix4Dmapper                                                                                                 & $\times$             & $\checkmark$    \\
                                           & Reality Capture                                                                                             & $\checkmark$         & $\checkmark$    \\
                                           & Autodesk Recap Photo                                                                                        & $\checkmark$         & $\checkmark$    \\ \hline
      \end{tabular}
      \begin{tablenotes}
        \footnotesize
        \item[1] charges for a complied installer for whom is difficult to install from source code
        \item[2] does not provide official documentation about making a geo-referenced GeoTiff file for common aerial products like \gls{dom} and \gls{dsm}
      \end{tablenotes}
      \end{threeparttable}
    }
    \end{center}
  % \end{adjustwidth}
\end{table}