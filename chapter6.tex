\chapter{\added{General conclusion}}

\added{This is the first study that combined the advances of aerial (high throughput) and close-range (high quality) surveys in plant phenotyping applications to field-grown crops. This Ph.D. thesis aimed to improve the performance of 3D-based plant phenotyping on field-grown broccoli as a representative of row-planted crops having harvestable organs on the top of canopy, using only low-cost \gls{rgb} cameras. Using \gls{uav} for aerial surveys and photogrammetry allows for the efficient acquisition of 2D field maps and 3D models of the crop canopy for the entire farmland. However, due to limitations in survey efficiency and wind blurring caused by propellers, the \gls{uav} cannot fly too close to plants, resulting in inadequate resolution and quality for directly analyzing the broccoli heads at the organ level. This thesis attempted to fuse the close-range and aerial 3D phenotype data, as well as some latest machine learning and deep learning techniques, to accurately and efficiently obtain the position and size of broccoli heads in the field. Furthermore, this thesis also provided a better 3D virtual visualization for those broccoli heads in the field and builds a foundation for digital twins and virtual farmland for smart agriculture.}

\added{In the second chapter, we have implemented an almost fully automated plant phenotyping pipeline based on the 3D reconstruction of broccoli heads. The described workflow has been designed to minimize the effects of limited perspective on the 3D model quality and completion. By using the dual-rotation of the object and the dual deep learning segmentation, our proposed workflow performed well on over 180 broccoli heads during the flowering season. The accuracy of pipeline calculated head size traits was validated by simple manual measurements using standard agricultural practices. The results suggest that our pipeline offers a great opportunity for high-throughput 3D phenotyping applications on the solid and enclosed plant organs (e.g. oranges, potatoes, cauliflowers, and sweet potatoes), in which size is directly related to harvest timing and profitability.}

\added{The third chapter aimed to adapt the \gls{uav}-based aerial sensing technology to the monitoring of spacial growth variation of field-grown crops. By using aerial photogrammetry and \gls{mldl}, we developed a system for estimating and predicting the head size of whole broccoli with high accuracy. This \gls{uav}-based prediction system is based on several technical improvements and requires minimal labor and computational costs. Therefore, it could be applied to support broccoli farming, and with modifications, to a variety of similar vegetables (i.e., cabbage, cauliflower, artichoke, and lettuce). Because our developed pipeline uses a simple sensor, not a complex integration of multiple sensors, it would be more applicable and offers user-friendliness for economically and socially disadvantaged rural regions, and it has the potential to be widely adopted by vegetable farmers worldwide.}

\added{The work in the fourth chapter aimed to develop a 3D virtual visualization technique that can fuse high-quality head 3D models (from Chapter 2) into low-quality full-field canopy models (from Chapter 3). By using piecewise affine transformation, \gls{automl} calibration model, and template matching and transformation, a 3D virtual visualization system was developed to visualize the broccoli head sizes at their field positions. The statistical analysis supports the improvements of the proposed method in geographical locations and broccoli morphological traits. The proposed 3D visualization system offers a great opportunity for virtual farmland and digital twin technology, which can provide a more intuitive feeling of growth status compared to numeric statistical values.}

\added{Overall, the results of these three research chapters showed that the proposed pipelines including active learning, deep learning, backward projection, auto machine learning, template matching, and data fusion, led to improved performance in the tested broccoli fields from 2020 to 2022. The results and statistical analysis concluded that the research objectives have been achieved and all the source codes are published on GitHub for replication and any usage purposes, we can conclude that the research has made a positive contribution to the 3D-based plant phenotyping and precision agriculture for broccoli farmlands.}