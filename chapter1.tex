\chapter{General Introduction}

\section{Research background}

Honbun honbun, honbun honbun \citep{guo_deep_2019, zhao_crop_2019}. 


\section{Objective of this study}

\begin{enumerate}
    \item To develop an almost-automatic 3D reconstruction workflow that can obtain the integrate and high-quality 3D models of destructively sampled broccoli heads.
    \item To develop an unsupervised phenotyping workflow that can automatically segment the broccoli crown part from Objective 1, and calculate several 1D to 3D morphological traits.
    \item To develop an improved workflow for UAV-based 3D reconstruction broccoli canopy using the 3D to 2D projection and labor-saving deep learning technique, which can obtain better 2D morphological traits of broccoli heads in complex outdoor conditions.
    \item To combine the strengths of the UAV-based pipeline (high throughput but low quality) and the destructive-based pipeline (high quality but low throughput). Using the auto machine learning regression and the template matching to recover the 3D traits from UAV-based 2D traits for all broccolis in the field.

\end{enumerate}


\section{Outline of this study}

Chapter 1 is an overview of the study's background information, relative studies, and objectives

Chapter 2 develops and validates the 3D phenotyping pipeline for destructively sampled broccoli heads using the photogrammetry technique, which includes obtaining high-quality plant 3D models and calculating the 3D traits.


Chapter 3 develops and validates the 3D phenotyping pipeline for \gls{uav} sensing on broccoli canopy, includes the 3D to 2D projection pipeline to improve deep-learning-based phenotyping

Chapter 4 tests the idea of cross-scale assimilation of broccoli based on the pipeline built in Chapter 2 and Chapter 3.

Chapter 5 summarizes the general conclusions of this study, and also discusses the research prospects in the future.