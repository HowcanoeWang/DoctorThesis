\chapter{General Introduction}

\section{Research background}

Honbun honbun, honbun honbun \citep{guo_deep_2019, zhao_crop_2019}. 


\section{Objective of this study}

\begin{enumerate}
    \item To develop indoor and outdoor plant phenotyping pipelines that are feasible on various kinds of crops and fields. 
    \item To extract canopy-level and organ-level morphological traits, and validate by field measurements.
    \item To test the feasibility of cross-scale assimilation on broccoli head.
    \item To develop a realistic procedural maize model whose structure is controlled by morphological traits and indoor scanned maize model.
\end{enumerate}


\section{Outline of this study}

Chapter 1 is an overview of the study's background information, relative studies, and objectives

Chapter 2 develops and validates the small-scale indoor 3D phenotyping pipeline, including obtaining and analyzing plant 3D information

Chapter 3 develops and validates the large-scale outdoor 3D phenotyping pipeline using drones equipped with commercial-level RGB cameras.

Chapter 4 tests the idea of cross-scale assimilation of broccoli and maize based on the pipeline built in Chapter 2 and Chapter 3.

Chapter 5 summarizes the general conclusions of this study, and also discusses the research prospects in the future.