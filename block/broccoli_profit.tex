This section has been modified and submitted to the ``Plant Phenomics"

\begin{center}
  \noindent
  \fbox{
    \begin{minipage}{0.95\textwidth}
    
      \begin{center}
        \textbf{Drone-based harvest data prediction can reduce on-farm food loss and improve farmer income}
      \end{center}

      \noindent Haozhou Wang$^{1}$, Tang Li$^{1}$, Erika Nishida$^{1}$, Yoichiro Kato$^{1}$, Yuya Fukano$^{2\star}$, and Wei Guo$^{1,\star}$
    
      \noindent $^{1}$ Graduate School of Agricultural and Life Sciences, The University of Tokyo, Tokyo, Japan
      
      \noindent $^{2}$ Graduate School of Horticulture, Chiba University, Chiba, Japan
      
      \noindent $^{\star}$ Corresponding authors

      \begin{spacing}{1.5}
      \textbf{Abstract}
      \end{spacing}
      
      On-farm food loss (i.e., grade-out vegetables) is a difficult challenge in sustainable agricultural systems. The simplest method to reduce the number of grade-out vegetables is to monitor and predict the size of all individuals in the vegetable field and determine the optimal harvest date. Here, we developed a full pipeline to accurately estimate and predict every broccoli head size (n $>$ 3000) automatically and nondestructively using drone remote sensing and image analysis and predicted the optimal harvesting date. Two years of field experiments revealed that our pipeline successfully estimated and predicted the head size of all broccolis with high accuracy. We successfully predicted the optimal harvest date and found that a deviation of only 1‒2 d from that date can significantly increase grade-out and reduce farmer profits. This is an unequivocal demonstration of the utility of these approaches to economic crop optimization and minimization of food losses. A short video that summarized this study can be found here: \url{https://youtu.be/SYuOCVqgtrU}

      \vspace{5mm}
      \textbf{keywords}: unmanned aerial vehicle (UAV), time-series analysis, deep learning, YOLO, BiSeNet

    \end{minipage}
  }
\end{center}