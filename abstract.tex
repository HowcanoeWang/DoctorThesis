\begin{eabstract}
  To ensure a sustainable increase in crop productivity for global food security, it is essential to improve plant phenotyping efficiency and accuracy through novel technologies. To date, measuring various crop traits such as aboveground morphology are time-consuming, laborious and sometimes inaccurate, especially in regards to three‐dimensional (3D) structures of the plant. Utlizing 3D phenotyping technique, which involves remote sensing and 3D reconstruction, is considered a powerful tool for this purpose. The technique entails obtaining and analyzing plant 3D information efficiently. However, applying the 3D phenotyping workflow directly in the outdoor field, where most agricultural production activities are conducted, is still challenging. For instance, blurring caused by wind and data loss due to canopy occlusion can reduce the accuracy of outdoor plant phenotyping by remote sensing. On the other hand, conducting 3D reconstruction in stable indoor environments can ensure high model quality but sacrifices throughput (processing speed) and requires destructive sampling, making it unsuitable to be applied to all plants in an outdoor field.
  
  To address the challenges mentioned above, the objective of this study was to improve the accuracy and throughput of 3D plant phenotyping by combining indoor and outdoor scale. The structure of the thesis is designed as follows: 1) introduction and literature review; 2) development of a 3D plant phenotyping pipeline for extracting morphological traits of broccoli heads, sweet potatoes, and maize organs in indoor environments; 3) development of a 3D plant phenotyping pipeline for extracting canopy-level and organ-level morphological traits in outdoor fields; 4) improvement of outdoor model quality by assimilating indoor data for broccoli and creating virtual plant model for maize; and 5) general discussion and conclusions.
  
 \paragraph{Indoor small-scale 3D plant phenotyping pipeline}

  The objective of Chapter 2 was to develop and validate the small-scale indoor 3D phenotyping pipeline, including obtaining and analyzing plant 3D information.

  For the methodology parts, first of all, two types of devices were designed to collect plant image data. One device was for an indoor environment with artificial light, a pure-color background curtain, and an automatic rotation platform. The other device was for greenhouse environment with natural lighting, auto-detectable markers and a handheld camera. Then, different computer vision and machine learning algorithms were used to extract plant regions from images, depending on the image complexity. The EasyDCP batch processing toolkit, integrated into the commercial-level SfM software (Agisoft Metashape), was developed to generate high-quality 3D models of plants. Finally, several algorithms were developed and applied to extract morphological traits

  As a result, the feasibility of the proposed method was tested at both the organ and plant levels using broccoli flowers, sweet potatoes, four cultivars of container weeds, and two cultivars of container maize. The height of the weeds and the long and short axes of broccoli flower were validated by manual measurement, while the commercial PlantEye laser scanning system validated the projected leaf area of weeds. For complex structures such as maize, the entire plant is first segmented into its individual leaves and stems using a published tool. Then, leaf inclination angle and leaf length were calculated and validated with manual measurement in 3D software. 

 \paragraph{Outdoor large-scale 3D plant phenotyping pipeline}

  The objective of Chapter 3 was to develop and validate the large-scale outdoor 3D phenotyping pipeline using drones equipped with commercial-level RGB cameras.

  For the methodology parts, the batch processing workflow was developed to obtain time-series 3D information on field plants by drones. For the canopy-level morphological traits, the canopy height and volume were extracted, taking the maize field as an example. Then a trained machine learning regression model was used to calculate biomass. For the organ-level morphological traits, broccoli heads were used as an example. To overcome challenges in distinguishing these small objects with high color similarity to the background, a simple detection task was performed at an early growth stage to obtain each broccoli’s geo-position and guide segmentation tasks later on. The head segmentation task was performed on the original UAV images, which had higher quality than reconstruction products. Its results were then linked back to the corresponding broccolis using the backward projection technique by our developed EasyIDP package. The weight of the broccoli head was also derived by a machine learning algorithm from the obtained morphological traits.

  As a result, for the canopy-level application, the height and biomass of maize canopy were first validated by destructive ground measurements, and their changes throughout the growing season were also traced by time-series data. Then, for the organ-level application, the feasibility of backward projection using the EasyIDP package was tested on six different crop fields (soybean, sugar beet, wheat, maize, orchard, and lotus), and its accuracy was validated by the lotus field with clear plot boundaries. Based on this, the size and predicted weight of broccoli head were calculated and validated by destructive field measurement.

 \paragraph{Cross indoor and outdoor scale data assimilation}

  The objective of Chapter 4 was to test the idea of cross-scale assimilation of broccoli and maize based on the pipeline built in Chapter 2 and Chapter 3.

  For the broccoli head case, around 200 broccoli heads paired data between outdoor and indoor throughout the growing season were collected by destructive sampling. To improve the accuracy of in-field traits affected by leaf occlusion, a machine learning model was trained to calculate indoor 3D traits (Cp. 2) using outdoor 2D traits (Cp. 3). To fix the missing parts of outdoor broccoli heads, we utilized the destructively sampled broccoli heads as a template database and the closest size template is transformed to fit back into the field canopy data. The accuracy was also validated by the field-measured broccoli heads.

  Due to the variation in maize architecture, the procedural maize model was developed at the current stage. This model allows for dynamic control of architecture by adjusting parameters such as leaf inclination angles and leaf root position. Referencing the maize model and traits obtained in Cp. 2, a large amount of realistic virtual maize plants were simulated. Based on this, we can attempt to recover the complex maize canopy architecture in the near future.

  \paragraph{In conclusion}
  
  this study showed that (1) the proposed indoor and outdoor plant phenotyping pipelines were feasible on various kinds of crops and fields, (2) some extracted morphological traits showed a high correlation with manual field measurement, (3) the cross-scale assimilation on broccoli showed the ability to fix missing parts and obtain advanced 3D traits, and (4) the proposed maize procedural model built the foundation for future architecture recovery and digital farmland.
  
\end{eabstract}

% \begin{jabstract}
%   概要、概要概要概要概要概要概要概要概要概要概要概要概要概要概要概要概要概要
% \end{jabstract}